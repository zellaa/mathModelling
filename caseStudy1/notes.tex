\documentclass[a4paper,openany,nobib]{tufte-book}
%define colours
\usepackage{xcolor}
\usepackage{amsmath}
\definecolor{tred}{HTML}{450000}
\definecolor{tgray}{HTML}{161616}
%%pagestyle
\usepackage[protrusion=true,expansion]{microtype}
\usepackage{booktabs}
\usepackage{fancyvrb}
\fvset{fontsize=\normalsize}
\fancypagestyle{plain}{}
%%references
%%final styling
\color{tgray}
\colorlet{darkgray}{tred}
\begin{document}
\thispagestyle{empty}
\chapter{Case Study 1: SIR Models} 
\setcounter{page}{1}
\section{Session 1:}%
\label{sec:Session 1:}

There are 3 components to consider:
\begin{enumerate}
	\item Susceptible $S(t)$ - \textit{those who can get the disease}
	\item Infectious $I(t)$ - \textit{those who have the disease}
	\item Recovered $R(t)$ - \textit{those who had the disease and no longer do}
\end{enumerate}
People in these catagories move from one another via the routine:
\begin{equation*}
	S \xrightarrow{\beta I} I  \xrightarrow{\gamma} R
\end{equation*}
To model this we use the \textbf{Law of Mass Action} i.e. the idea that \textbf{the rate of change is proportional to the product of reactants involved in the equation}.

For example we have:
\begin{align*}
	A+ B &\xrightarrow{k} C\\
	\frac{d[A]}{dt} &= -k \left[ A \right]\left[ B \right] 
\end{align*}

In our context we thus have the equations:
\begin{align}
	  \label{eq:eqn1}
	\frac{dS}{dt} &=  - \beta SI\\ 
	  \label{eq:eqn2} 
	\frac{dI}{dt} &= \beta S I - \gamma I\\
	  \label{eq:eqn3}
	\frac{dR}{dt} &= \gamma I
\end{align}
\marginnote[-2cm]{Here $\beta$ is the infection rate (per time, per infective), $\gamma$ is the rate at which the infected are removed.}
\marginnote{Note that:
\begin{align*}
	S(0)&= S_0\\
	I(0) &= I_0\\
	R(0) &= 0
\end{align*}
}

We have several assumptions inherent to our basic model:
\begin{enumerate}
	\item[A1] No natural births or deaths,
	\item[A2] No spatial effects,
	\item[A3] System is well mixed,
	\item[A4] Infection and ``fixing'' (i.e. whatever process causes someone to no longer be infected) occur instantaneously,
\end{enumerate}

We can add equations \eqref{eq:eqn1} - \eqref{eq:eqn3} to see that $S+I+R=S_0+I_0 :=N$, where $N$ is a constant which is the total population.
\marginnote{Also note that \eqref{eq:eqn3} is able to be decoupled.}

We note now that \begin{equation}
	\frac{dI}{dt} = \beta \left( S-\rho \right)I 
\end{equation}
where $\rho = \gamma/\beta$.
\marginnote{Epidemics only occur if $S_0 > \rho$ at $t=0$.}
We now define an \textbf{epidemic} which describes the situation if, at $t=0$
\begin{equation*}
	\frac{dI}{dt}> 0.
\end{equation*}
We let $R_0 =S_0 / \rho = \beta S_0 / \gamma$ be the \textbf{basic reproduction number}; we have then that if $R_0 >1$, an epidemic will occur.

\subsection{To recap:}%
\label{sub:To recap:}

\begin{enumerate}
	\item $\beta S_0$ is the rate at which an infected person produces infections in a population of $S_0$ people susceptible to the disease.
	\item $1/\gamma$ is the time someone is infectious.
	\item An infected person will produce $\beta S_0 \times 1/\gamma$ new infected people.
\end{enumerate}
\section{Session 2:}%
\label{sec:Session 2:}
\textbf{Nondimensionalisation}
\marginnote{In our case we might define: 
\begin{enumerate}
	\item $S\rightarrow[S]\hat S$
	\item $I\rightarrow[I]\hat I$
	\item $t\rightarrow[t]\hat t$
\end{enumerate}
Where ``hatted'' variables have been scaled by the bracketed variables, i.e. $\hat S := S/[S]$.}
might be useful in order to better highlight important constants, and also puts the model into a framework in order to better focus on the size of variables.

For population models we tend to use $N$ as the scaling variable i.e. $\hat S := S/N, \hat I := I/N$.
We then have:
\begin{align*}
	\frac{N}{[t]} \frac{d\hat S}{d\hat t} &= -\beta N^2 \hat S \hat I\\
	\frac{N}{[t]} \frac{d\hat I}{d\hat t} &= \beta N^2 \hat S \hat I - \gamma N \hat I\\
\end{align*}
where $t = [t]\hat t$.
We might then scale $[t] = 1/\gamma$ e.g:
\begin{align*}
	\frac{d\hat S}{d\hat t} &= -\beta N \hat S \hat I \gamma\\
	 \frac{d\hat I}{d\hat t} &= \beta \frac{N}{\gamma}  \hat S \hat I -  \hat I
\end{align*}
Here $t$ has been scaled by the timescale of recovery, whereas if we had chosen $[t] = 1/\beta N$ it would represent the timescale of infection,
\begin{align}
	\label{eq:nd1}
	\frac{d\hat S}{d\hat t} &= -\hat S \hat I\\
	\label{eq:nd2}
	 \frac{d\hat I}{d\hat t} &= \hat S \hat I - \mu \hat I
\end{align}
where $\mu := \frac{\gamma}{\beta N}$ 

\subsection{Further Exploration}%
\label{sub:Further Exploration}

\begin{itemize}
	\item Include a vaccination rate which kicks in at time $T$, and is a function of the amount dead, infected, and susceptible etc
\end{itemize}
\end{document}
